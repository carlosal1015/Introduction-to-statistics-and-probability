\documentclass[a4paper]{exam}

\usepackage{mathtools}
\title{Laboratorio $1$}
\author{Oromion}
\begin{document}
\begin{questions}
\question Defina una matriz de nombre \texttt{AA} de tamaño $4\times4$ con
  números seleccionados por usted. Responda las siguientes preguntas, haciendo
  uso de los comandos de R.
  \begin{parts}
  \part ¿Qué tipo de dato es \texttt{AA}?
    
  \part Puede modificar \texttt{AA} y crear otros tipos de datos.
  \end{parts}
  
\question Defina un factor en base alguna variable estadística que desea
  caracterizar usted.
  \begin{parts}
  \part Unir las matrices, en base a las formas señaaladas.
  \part Repetir esto creando \texttt{A1}, \texttt{A2} y \texttt{A3} para todos
    los tipos de datos posibles.
  \end{parts}
  
\question Construya un cuadro de datos llamado \texttt{notas} con la siguiente
  información:
  \begin{table}[ht!]
    \centering
    \begin{tabular}{lccc}
      & Examen $1$ & Examen $2$ & Tareas \\
      Antonio & $7$ & $6$ & $7$ \\
      Carlos & $8$ & $8$ & $9$
    \end{tabular}
    \caption{Notas de los alumnos}
    \label{tab:grades}
  \end{table}
  Con las herramientas aprendidas hasta el momento.
  \begin{parts}
  \part Muestre las notas de los exámenes $1$, $2$ y $3$ del alumno Antonio.
    Haga lo mismo para el resto de los alumnos.
  \part Muestre las notas del examen $1$ de los alumnos. Haga los mismo para el
    resto de los exámenes.
  \part Muestre la nota máxima de cada alumno.
  \part Muestre la nota promedio de cada alumno.
  \end{parts}
  
\question Cargue el marco de datos \texttt{mtcars} brindado por R.
  \begin{parts}
  \part Describa que ofrece dicha data.
  \part Usando las herramientas presentadas hasta el momento, explote todo el
    potencial de la data \texttt{mtcars}.
  \end{parts}
\end{questions}
\end{document}