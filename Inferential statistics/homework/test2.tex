\documentclass[a4paper]{scrreprt}
\KOMAoptions{BCOR=8.25mm}
\usepackage{amsthm}
\usepackage{mathtools}
\usepackage{dsfont}

\newtheorem{definition}{Definición}
\newtheorem{example}{Ejemplo}
\newtheorem{proposition}{Proposición}

\providecommand{\inner}[2]{#1,#2}
\begin{document}
% Hay infinitas normas en $\mathds{R}^{n}$.
% Algunas normas se prestan para los cálculos.
% El límite no depende de una norma.
% Muchas normas provienen de un producto interno.
% También hay normas que no provienen de ningún producto interno.
\section{El producto interno sobre $\mathds{R}^{n}$}
En $\mathds{R}^{n}$ con $n\geq2$, la aplicación $\langle,\rangle\colon\mathds{R}^{n}\times\mathds{R}^{n}\rightarrow\mathds{R}$, $\langle,\rangle\left(x,y\right)=\langle x,y\rangle$ es un producto interno real sobre $\mathds{R}^{n}$ se cumple:
\begin{enumerate}
	\item $\langle x,y\rangle=\langle x,y\rangle$.
	\item $\inner{x}{y}$
\end{enumerate}
\end{document}