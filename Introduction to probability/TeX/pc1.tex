\documentclass{exam}
\usepackage[spanish,es-nodecimaldot]{babel} \spanishdatedel
\usepackage{float}

\begin{document}
	\begin{questions}
	\question El aplazamiento en el cobro de las \emph{últimas cien ventas facturadas} por un establecimiento se había agrupado en cuatro intervalos, recordándose solo los siguientes datos de la distribución:

		\begin{parts}
			\part El primer intervalo tiene seis semanas como extremo superior, una frecuencia relativa de $0,2$ y una amplitud de cuatro semanas.
			\part En el segundo intervalo se acumulan $60$ ventas.
			\part Las marcas de clase del segundo y cuarto intervalos son $8$ y $50$ semanas, respectivamente.
			\part El tercer intervalo presenta una frecuencia de $30$ ventas y una amplitud de $30$ semanas. Con esta información reconstruya la distribución de frecuencias y represente el histograma correspondiente.
		\end{parts}
	
	\question En un convenio laboral se acuerda subir un $10$\% el volumen total de salarios. Un empresario con $250$ empleados les paga un total de \$$200000$ mensuales. Se sabe además que dicha variable presenta una desviación típica de \$$120$. La subida del $10$\% representa para el empresario un incremento total de las nóminas del personal de \$$20000$ mensuales. Dicha subida puede ser:

	\begin{parts}
		\part Se aumenta el sueldo de cada empleada en un $10$\%.
		\part Se reparte \$$20000$ entre los $250$ empleados, a todos por igual.
	\end{parts}
	
	¿Qué alternativa conduce reducir las diferencias salariales?
	
	\begin{table}[H]
		\centering
		\begin{tabular}{|c|c|}
			\hline
			Tamaño de los hogares $(X_i)$ & Número de hogares $(n_i)$\\
			\hline
			1 & 40\\
			\hline
			2 & 70\\
			\hline
			3 & 110\\
			\hline
			4 & 90\\
			\hline
			5 & 48\\
			\hline
			6 & 42\\
			\hline
			7 & 40\\
			\hline
			8 & 35\\
			\hline
			9 & 20\\
			\hline
			10 & 5\\
			\hline
			Total & 500\\
			\hline
		\end{tabular}
	\caption{}
	\end{table}

	\question En las siguientes situaciones, ¿cuál de los estadísticos media muestral o mediana muestral piensas que es más informativo?
	\begin{parts}
		\part Para analizar si se debe cerrar una línea de autobús entre Lima y Chiclayo, un ejecutivo ha recopilado el número de viajeros en una muestra de días.
		\part Para comparar a los estudiante universitarios actuales con los de años anteriores, se consultan muestras de las calificaciones obtenidas en los exámenes de acceso a la universidad durante varios años.
		\part El abogado defensor de un proceso judicial con jurado popular está analizando las puntuaciones de un test de inteligencia (IQ) obtenida por los miembros del jurado.
		\part Tú has comprado su casa hace seis años en una pequeña comunidad por un precio de \$$105000$, que coincidía con el precio medio y mediano de todas las casas que se vendieron aquel año en la dicha comunidad. Sin embargo, en los dos últimos años, se han construido varias casas nuevas mucho más caras que las anteriores. Para obtener una idea del valor actual de su casas, decides analizar los precios de venta de las casas vendidas recientemente en su comunidad.
	\end{parts}

Además, calcula el coeficiente de correlación muestral para los pares de datos anteriores.
	\end{questions}
\begin{flushright}
	\hfill Los profesores (César Lara, Fernando Zamudio)\\
	UNI, \today
\end{flushright}
\end{document}