\documentclass{exam}
\usepackage{float}
\begin{document}
	\begin{questions}
		\question La población de una determinada provincia durante los años que se indica fue la siguiente
		\begin{table}[H]
			\centering
			\begin{tabular}{|c|c|}
			\hline
			Año & Población\\
			\hline
			1995 & 375.1\\
			\hline
			1996 & 385.7\\
			\hline
			1997 & 390.6\\
			\hline
			1998 & 410.5\\
			\hline
			1999 & 430.3\\
			\hline
			2000 & 450.7\\
			\hline
			\end{tabular}
		\caption{Población de cierta ciudad durante los años 1995 y 2000}
		\end{table}
	Obtener la tasa de crecimiento anual.
	\question Obtener el salario mensual más frecuente, expresado en euros, para la siguiente distribución:
	\begin{table}[H]
		\centering
		\begin{tabular}{|c|c|}
			\hline
			Salarios & Asalariados\\
			\hline
			Menos de 500 & 50\\
			\hline
			500 - 900 & 70\\
			\hline
			900 - 1200 & 120\\
			\hline
			1200 - 1800 & 100\\
			\hline
			1800 - 27000 & 50\\
			\hline
			2700 - 5000 & 20\\
			\hline
		\end{tabular}
	\caption{Cantidad de asalariados en función de su salario.}
	\end{table}
\begin{parts}
	\part El salario mínimo del 25\% de los asalariados que más ganan.
	\part El salario máximo del 40\% de los que menos ganan.
	\part El porcentaje de asalariados con salarios comprendidos entre la centila 25 y la decila 9 y la varianza.
\end{parts}
	\question Los siguientes conjuntos de datos representan las puntuaciones obtenidas por 40 estudiantes de un curso de probabilidad en un test IQ en una determinada escuela matemática:
	\begin{center}
		144,122,103,118,99,105,134,125,117,106\\
		109,104,111,127,133,111,117,103,120,98\\
		100,130,141,119,128,106,109,115,113,121\\
		100,130,125,117,119,113,114,104,108,110,112
	\end{center}
\begin{parts}
	\part Presente este conjunto de datos en un histograma de frecuencias.
	\part ¿Qué intervalo de clase contiene el mayor número de valores de datos?
	\part ¿Existe el mismo número de datos en cada uno de los intervalos de clase?
	\part ¿El histograma parece aproximadamente simétrico?
\end{parts}
\question La distribución de los hogares de un determinado barrio, según el tamaño de los mismos, es la que muestra la tabla 3. Calcule el número medio de personas por hogar, el tipo de hogar con más frecuencia. Si solo hubiera plazas de aparcamiento para el 50\% de los hogares y éstas se asignaran a las de mayor tamaño, ¿a partir de qué tamaño de hogar se le asignarían plaza de garaje? Si en otro barrio el coeficiente de variación es 1, ¿en cuál de los barrios la media resulta más representativa?

\begin{table}[H]
	\centering
	\begin{tabular}{|c|c|}
		\hline
		Tamaño de hogares & Número de hogares\\
		 \hline
		 1 & 40\\
		 \hline
		 2 & 70\\
		 \hline
		 3 & 110\\
		 \hline
		 4 & 90\\
		 \hline
		 5 & 48\\
		 \hline
		 6 & 42\\
		 \hline
		 7 & 40\\
		 \hline
		 8 & 35\\
		 \hline
		 9 & 20\\
		 \hline
		 10 & 5\\
		 \hline
		 Total & 500\\
		 \hline
	\end{tabular}
\end{table}

\question Un conjunto de datos puntuales se dividió en 8 clases, todas de tamaño 3 (en las unidades de los datos). Después se determinaron las frecuencias de cada clase y se construyo una tabla de frecuencias. Sin embargo, ciertas entradas de esta tabla se perdieron. Supongamos que la parte de la tabla de frecuencia que se conservó es la siguiente
\begin{table}[H]
	\centering
	\begin{tabular}{|c|c|c|}
		\hline
		Intervalo de clase & Frecuencia & Frecuencia relativa\\
		\hline
	\end{tabular}
\end{table}
Completa los valores perdidos de la tabla y dibuja un histograma de frecuencias relativas.
\question Para determinar la relación entre la temperatura que hay al mediodía (medida en grados Celsius) y el número de piezas defectuosas producidas dicho día, una compañía registró los datos siguientes a 22 días laborales.
\begin{table}[H]
	\centering
	\begin{tabular}{|c|c|c|}
		\hline
		Temperatura & Número de piezas defectuosas\\
		\hline
		24.2 & 25\\
		\hline
		22.7 &  31\\
		\hline
		30.5 & 36\\
		\hline
		28.6 & 33\\
		\hline
		25.5 & 19\\
		\hline
		32.0 & 24\\
		\hline
		28.6 & 27\\
		\hline
		26.5 & 25\\
		\hline
		25.3 & 16\\
		\hline
		26.0 & 14\\
		\hline
		24.4 & 22\\
		\hline
		24.8 & 23\\
		\hline
		20.6 & 20\\
		\hline
		25.1 & 25\\
		\hline
		21.4 & 25\\
		\hline
		23.7 & 23\\
		\hline
		23.9 & 27\\
		\hline
		25.2 & 30\\
		\hline
		27.4 & 33\\
		\hline
		28.3 & 32\\
		\hline
		28.8 & 35\\
		\hline
		26.6 & 24\\
		\hline
	\end{tabular}
\end{table}
\begin{parts}
	\part Dibuja un diagrama de dispersión.
	\part ¿Qué se puede concluir a partir del diagrama anterior?
	\part Si la temperatura al mediodía fuese de 24 C, ¿qué se podría conjeturar sobre el número de piezas defectuosas que se vayan a producir al día siguiente?
\end{parts}
	\question Encuentre el percentil muestral de orden 90\% del siguiente conjunto de datos:
	\begin{center}
		75,33,51,21,46,98,103,88,35,22,29,73,37,101,121,144,133,52,54,63,21,7.
	\end{center}
\question Los cuartiles de un extenso conjunto de datos son los siguientes:
\begin{center}
	Primer cuartil = 35\\
	Segundo cuartil = 47\\
	Tercer cuartil = 66
\end{center}
\begin{parts}
	\part Indica un intervalo que contenga aproximadamente un 50\% de los datos.
	\part Determina un valor que aproximadamente sea mayor que un 50\% de los datos.
	\part Determina un valor para el que aproximadamente un 25\% de los datos sean mayores que él.
\end{parts}
\question Suponga que la media muestral de un conjunto de 10 datos puntuales es $\bar{x}=20$.
\begin{parts}
	\part Si se descubre que se ha leído incorrectamente un dato con valor 15 y que se le ha dado el valor 13. ¿Cuál será el valor revisado de la media muestral?
	\part Si existiera un dato adicional con valor 22, ¿aumentaría o disminuiría el valor de $\bar{x}$?
	\part Con los datos originales, ¿cuál sería el nuevo valor de $\bar{x}$ del apartado anterior?
\end{parts}
\question Supongamos que se disponen de dos muestras distintas, de tamaños $n_1$ y $n_2$. Si la media muestral de la primera muestra es $\bar{x}_1$ y la segunda muestra es $\bar{x}_2$. ¿Cuál es la media de la muestra conjunta, de tamaño $n_1+n_2$?
\question La mediana de un conjunto de datos simétricos es igual a 40 y su tercer cuartil es igual a 55. ¿Cuál es el valor del primer cuartil?
\question Suponga que deseas descubrir el salario del vicepresidente de un banco, al que acabas de conocer. Si pretendes tener la mayor probabilidad de acertar a menos de 1000 soles, ¿le gustaría conocer la media muestral, la mediana muestral o la moda muestral de los salarios de los vicepresidentes de bancos?
\question Varios corredores utilizan una pista de atletismo de un cuarto de milla de longitud. En una muestra de 17 corredores, 1 corrió dos vueltas, 4 corrieron cuatro vueltas, 5 corrieron seis vueltas, 6 corrieron ocho vueltas y 1 corrió doce vueltas.
\begin{parts}
	\part ¿Cuál es la moda muestral del número de vueltas que han hecho estos corredores?
	\part ¿Cuál es la moda muestral de las distancias en millas recorridas por los corredores?
\end{parts}
\question Un individuo que necesitaba asegurar su coche, preguntó cuáles eran las cuotas para idénticas coberturas en 10 compañías de seguro. Obtuvo los siguientes valores (correspondientes a las cuotas anuales, en dólares)
\begin{center}
	720,880,630,590,1140,908,677,720,1260,800
\end{center}
Encuentre:
\begin{parts}
	\part la media muestral
	\part la mediana muestral
	\part la desviación muestral
\end{parts}
\question Si $s$ es la desviación típica muestral de los datos $x_i$, $i=1,2,\ldots,n$. ¿Cuál es la desviación típica muestral de $ax_i+b$,  $i=1,2,\ldots,n$? En este problema $a$ y $b$ son constantes dadas.
\question En las siguientes situaciones, ¿cuál de los estadísticos media muestral o mediana muestral piensas que es más informativo?
\begin{parts}
	\part Para analizar si se debe cerrar una línea de autobús entre Lima y Chiclayo, un ejecutivo ha recopilado el número de viajeros en una muestra de días.
	\part Para comparar a los estudiantes universitarios actuales con los de los años anteriores, se consultan muestras de las calificaciones obtenidas en los exámenes de acceso a la universidad durante varios años.
	\part El abogado defensor de un proceso judicial con jurado popular está analizando las puntuaciones de un test de inteligencia (IQ) obtenida por los miembros del jurado.
	\part Ha comprado su casa hace seis años en una pequeña comunidad por un precio de 105000 dólares, que coincidía con el precio medio y mediano de todas las casas que se vendieron aquel año en dicha comunidad. Sin embargo, en los dos últimos años, se han construido varias casas nuevas mucho más caras que las anteriores. Para obtener una idea del valor actual de su casa, decide analizar los precios de venta de las casas vendidas recientemente en su comunidad.
	% TODO Agregar dibujo
\end{parts}
	\question La siguiente tabla muestra el número de médicos y dentistas que había en Japón en los años ares comprendidos entre 1984 y 2000.
	\begin{table}[H]
		\centering
		\begin{tabular}{|c|c|c|}
			\hline
			- & Médicos & Dentistas\\
			\hline
			1984 & 173452 & 61283\\
			\hline
			1986 & 183129 & 64904\\
			\hline
			1988 & 193682 & 68692\\
			\hline
			1990 & 203797 & 72087\\
			\hline
			1992 & 211498 & 75628\\
			\hline
			1994 & 220853 & 79091\\
			\hline
			1996 & 230297 & 83403\\
			\hline
			1998 & 236933 & 85669\\
			\hline
			2000 & 243201 & 88410\\
			\hline
		\end{tabular}
	\end{table}
\begin{parts}
	\part Determina la varianza muestral del número de médicos en los años citados.
		\part Determina la varianza muestral del número de dentistas en los años citados.
\end{parts}
\question Considere los dos siguientes conjuntos de datos
\begin{center}
	A: 4,5,0,5,1,5,0,10,5,2\quad B: 0,4,0,1,9,0,10,9,5
\end{center}
\begin{parts}
	\part Determine el rango de cada conjunto de datos.
	\part Calcule la desviación típica muestral de cada conjunto de datos.
	\part Determine el rango intercuartílico de cada conjunto de datos.
\end{parts}
\question Explica por qué el coeficiente de correlación muestral de los pares de datos
\begin{center}
	(121,360), (242,362), (363,364)
\end{center}
es el mismo que el de los pares
\begin{center}
	(1,0), (2,2), (3,4)
\end{center}
el cual, a su vez, coincide con el de los pares
\begin{center}
	(1,0), (2,1), (3,2)
\end{center}
Además, calcula el coeficiente de correlación muestral para los pares de datos anteriores.
	\end{questions}
\begin{flushright}
	\hfill Los profesores (César Lara, Fernando Zamudio)\\
	UNI, \today
\end{flushright}
\end{document}